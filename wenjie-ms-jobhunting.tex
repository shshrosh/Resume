\documentclass[margin,line]{res}

\usepackage{array}
\usepackage{enumitem}
\usepackage{color}
%\usepackage{tabularx}
\oddsidemargin -.5in
\evensidemargin -.5in
\textwidth=6.0in
\itemsep=0in
\parsep=0in
% if using pdflatex:
%\setlength{\pdfpagewidth}{\paperwidth}
%\setlength{\pdfpageheight}{\paperheight}

\newenvironment{list1}{
  \begin{list}{\ding{113}}{%
      \setlength{\itemsep}{0in}
      \setlength{\parsep}{0in} \setlength{\parskip}{0in}
      \setlength{\topsep}{0in} \setlength{\partopsep}{0in}
      \setlength{\leftmargin}{0.17in}}}{\end{list}}
\newenvironment{list2}{
  \begin{list}{$\bullet$}{%
      \setlength{\itemsep}{0in}
      \setlength{\parsep}{0in} \setlength{\parskip}{0in}
      \setlength{\topsep}{0in} \setlength{\partopsep}{0in}
      \setlength{\leftmargin}{0.2in}}}{\end{list}}


\begin{document}

\name{\LARGE Wenjie Hu \vspace*{.1in}}

\begin{resume}
\section{\sc Contact Information}
\vspace{.05in}
\begin{tabular}{@{}p{3.5in}p{2.3in}}
%Building 84/2051           &  \\
Microsoft Building 84/3349   & \hfill  Telephone: (814)-954-2868\\
Redmond, WA  & \hfill Email: \texttt{wenhu@microsoft.com}
\end{tabular}


%\section{\sc Brief\\ Description}
%I got my Bachelor, Master and Ph.D {\bf all in Computer Science} and I have solid background in software engineer. I have also worked in {\bf Deep Learning/Machine Learning} for several years. I am looking for a full time job as a software engineer, or a combination of both software engineer and data scientist.

\section{\sc Education}

{\bf The Pennsylvania State University}  \hfill  University Park, PA\\
{ {\bf Ph.D} in Computer Science and Engineering} \hfill 2016 \\
%Advisor: Dr. Guohong Cao
%Research interests: Mobile computing, mobile cloud, energy optimization for mobile device
\vspace{-0.25in}

{\bf Tsinghua University} \hfill Beijing, China\\
{{\bf M.S.} in Computer Science and Technology} \hfill 2010\\
%Advisor: Dr. Yong Cui
\vspace{-0.25in}

{\bf Tongji University} \hfill Shanghai, China \\
{{\bf B.E.} in Computer Science and Technology} \hfill 2007
\vspace{-0.1in}

%{\bf Shanghai University of Finance and Economics} \hfill Shanghai, China \\
%{{\bf Minor} in Business Administration} \hfill 2006

\section{\sc Working Experience}

{\bf Microsoft Corp. }   \\ %\hfill  Mar. 2016 - Present
{\bf MKG (Microsoft Knowledge and Growth) Data Sciences Team}   \\
\begin{itemize}
  \item {\bf Classification/Ranking} \\
  Deployed a \emph{Fast Tree} classification model to predict whether a potential customer will buy Microsoft products in next two months. \\ Conducted calibration to provide a meaningful score to the inside sellers, i.e., for 100 customers with score 80, around 80 will be converted after two months. \\
  Got a lift of 10-40\% over rule based model. %(classification problem/ranking problem).
  \item {\bf Regression} \\
  Shipped 12 \emph{Fast Tree} regression models to predict the customer life time value in 12 Microsoft product divisions. \\
  Used feature correlation/feature importance to select proper features.\\
  Designed a metric combining L1/L2/Pearson Correlation/Spearman Correction to select the best model. % (regression problem).
  \item {\bf Causal Effect Analysis} \\
  Analyzed the impact of Microsoft reward program/family program on the total revenue. \\
  Used \emph{Propensity Score} to remove bias when selecting counterfactual. \\
  Solved the problem to compare two groups of users when it is impossible to run A/B test. \\
  Published one paper and submitted one patent.
  \item {\bf Deep Learning}\\
  Rebuilt \emph{feed forward network/RNN (LSTM)} to predict users' spending. \\
  Designed \emph{parameter sweep} modules to tune the hyperparameters. \\
  Leveraged \emph{Autoencoder} to reduce the feature dimension.
  \item {\bf Time Series Prediction} \\
  Used \emph{STL/Holt-Winters/ARIMA} to predict a user's spending in the next year based on her spending in previous 2 years.
  \item {\bf Clustering} \\
  Divided users into clusters and then used the cluster id as features to predict the user's spending.
 
\end{itemize}

{\bf Jarvis Engineer Team}   \\
\begin{itemize}
  \item {\bf System Implementation}\\
   Designed and deployed APIs to collect data from different sources about a commercial customer and then show the information in a unified interface to inside sellers.
  \item {\bf NLP}\\
   Built a natural language generation (NLG) module to provide a brief summary of a customer.
\end{itemize}


\section{\sc Program Languages}
Java, Python, Scope, C/C++, C\#, R, javascript\\

%\section{\sc Skills}
%{\bf Languages:} C/C++ (10+ years), Java (8 years), Python, Matlab, javascript, R, SQL, HTML\\
%{\bf Operating Systems:} Linux, Android, Windows\\
%{\bf Tools:} Visual Studio, Eclipse, git, vi


\section{\sc Honors and Awards}
{\bf Microsoft Special Bonus Award} (only one in MKG data science team) \\
Microsoft \hfill 2018
\vspace{-0.1in}

{\bf Graduate Research Assistant Award} (highest honor for graduate student)\\
The Pennsylvania State University \hfill {2015}
\vspace{-0.1in}

{\bf AT\&T Graduate Fellowship} (only one each year)\\
AT\&T \hfill {2015}
\vspace{-0.1in}

{\bf Student Travel Grant}\\
IEEE INFOCOM 2015, ACM MobiHoc 2014, IEEE ICDCS 2013
\vspace{-0.1in}

%{\bf College of Engineering Fellowship}\\
%The Pennsylvania State University \hfill { 2010}
%\vspace{-0.1in}
%
%{\bf Guanghua Scholarship for Excellent Students} \\
%Tsinghua University \hfill  { 2008}
%\vspace{-0.1in}

%{\bf Kang-Shien Scholarship for Excellent New Student(only 1 in Computer Science)}\\
%Tsinghua University \hfill { 2007}
%\vspace{-0.1in}

%{\bf Best Graduate (TOP 5\%)} \\
%Tongji University \hfill  { 2007}
%
%\section{\sc Academic Experience}
%{\bf Research Assistant} \hfill Jan. 2011 - present\\
%Dept. of Computer Science and Engineering \hfill The Pennsylvania State University\\
%Research on mobile computing, energy optimization
%%\vspace*{.05in}
%%\begin{list2}
%% \item {\bf Mobile Video Streaming Module}: Added a video streaming module in VLC (an open source video player) to select the most energy efficient way to download videos based on the network conditions and user behaviors.
%% \item {\bf Traffic Offloading Framework}: Proposed a framework to offload network traffic to neighbors with better cellular signal to save energy; Developed a web browser and a photo uploader on top of the framework.
%% \item {\bf Measurement Tools}: Developed tools to measure the throughput of Bluetooth, WiFi Direct, WiFi and 3G/4G/LTE; Studied the power characteristics of these interfaces.
%% \item {\bf SmartPhoto}: Developed a photo app to record the phone's location, position, and camera focus when taking photos and automatically filter out photos covering a specific target.
%% \item {\bf Data Dissemination System}: Built a system on Android to let users share local files via Bluetooth when contacting each other.
%%\end{list2}
%
%{\bf Research Assistant} \hfill Jan. 2008 - Jul. 2010\\
%Dept. of Computer Science and Technology \hfill Tsinghua University\\
%%\vspace*{.05in}
%Research on MAC and routing protocols in Wireless Mesh Networks.
%
%\section{\sc Industrial Experience}
%
%{\bf Research Intern} \hfill Jun. 2014 - Aug. 2014\\
%Samsung Research America, San Jose, CA \\
%{\bf Mentor:} Dr. Chiu Ngo
%\vspace*{.05in}
%\begin{list2}
% \item Developed a vehicle communication system to spread emergency messages and improve the driving safety.
%\end{list2}


\section{\sc Publications}


%\vspace{-0.1in}
{\bf Conference Papers}
\vspace*{.05in}
\begin{enumerate} [leftmargin=5mm]% \itemsep -1pt % Reduce space between items
\item {\bf Wenjie Hu} and Guohong Cao, ``Energy-Aware CPU Frequency Scaling for Mobile Video Streaming'', {\em in Proceedings of the 37th IEEE International Conference on Distributed Computing Systems (ICDCS)}, 2017.
\item Yi Yang, Yeli Geng, Li Qiu, {\bf Wenjie Hu} and Guohong Cao, ``Context-Aware Task Offloading for Wearable Devices'', {\em in Proceedings of the 26th International Conference on Computer Communications and Networks (ICCCN)}, 2017.
\item Yibo Wu, Yi Wang, {\bf Wenjie Hu} and Guohong Cao, ``Resource-Aware Photo Crowdsourcing Through Disruption Tolerant Networks'', {\em in Proceedings of the 36th IEEE International Conference on Distributed Computing Systems (ICDCS)}, 2016. Acceptance Ratio: 17.6\%. %, Citation: 1.
\item {\bf Wenjie Hu} and Guohong Cao, ``Energy-Aware Video Streaming on Smartphones'', {\it in Proceedings of the 34th IEEE International Conference on Computer Communications (INFOCOM)}, 2015. Acceptance Ratio: 19.3\%.%, Citation: 6, Independent Citation: 6.
\item {\bf Wenjie Hu} and Chiu Ngo, ``LMAC: LTE-assisted MAC Protocol to Reduce Delay for Vehicle to Vehicle Communications'', {\em in IEEE International Conference on Communications (ICC)}, 2015. %Acceptance ratio: 30\%.
\item Yeli Geng, {\bf Wenjie Hu}, Yi Yang, Wei Gao and Guohong Cao, ``Energy-Efficient Computation Offloading in Cellular Networks'', {\it in Proceedings of the 23rd IEEE International Conference on Network Protocols (ICNP)}, 2015. Acceptance Ratio: 20.3\%.%, Citation: 2, Independent Citation: 2.
\item Xiao Sun, Zongqing Lu, {\bf Wenjie Hu} and Guohong Cao, ``SymDetector: Detecting Sound-Related Respiratory Symptoms Using Smartphones'', {\it in Proceedings of the ACM International Joint Conference on Pervasive and Ubiquitous Computing (UbiComp)}, 2015. Acceptance Ratio: 23.6\%.%, Citation: 6, Independent Citation: 6.
\item  {\bf Wenjie Hu} and Guohong Cao, ``Quality-Aware Traffic Offloading in Wireless Networks'', {\em in Proceedings of the 15th ACM International Symposium on Mobile Ad Hoc Networking and Computing (MobiHoc)}, 2014. Acceptance Ratio: 18.9\%.%, Citation: 10, Independent Citation: 7.
\item Yi Wang, {\bf Wenjie Hu}, Yibo Wu and Guohong Cao, ``SmartPhoto: A Resource-Aware Crowdsourcing Approach for Image Sensing with Smartphones'', {\em in Proceedings of the 15th ACM International Symposium on Mobile Ad Hoc Networking and Computing (MobiHoc)}, 2014. Acceptance Ratio: 18.9\%.%, Citation: 21, Independent Citation: 21.
\item {\bf Wenjie Hu} and Guohong Cao, ``Energy Optimization Through Traffic Aggregation in Wireless Networks'', {\em in Proceedings of the 33rd IEEE International Conference on Computer Communications (INFOCOM)}, 2014. Acceptance Ratio: 19.4\%.%, Citation: 16, Independent Citation: 12.
\item {\bf Wenjie Hu}, Guohong Cao, Srikanth V. Krishanamurthy, and Prasant Mohapatra, ``Mobility-Assisted Energy-Aware User Contact Detection in Mobile Social Networks'', {\em in Proceedings of the 33rd IEEE International Conference on Distributed Computing Systems (ICDCS)}, 2013. Acceptance Ratio: 13\%.%, Citation: 18, Independent Citation: 13.
\item Yong Cui, {\bf Wenjie Hu}, Sasu Tarkoma, and Antti Yla-Jaaski, ``Probabilistic Routing for Multiple Flows in Wireless Multi-hop Networks'', {\em in Proceedings of the 34th IEEE Conference on Local Computer Networks (LCN)}, 2009. % Citation: 3, Independent Citation: 3. %(short paper)
\end{enumerate}

{\bf Journal Papers}
\vspace*{.05in}
\begin{enumerate} [leftmargin=5mm]%\itemsep -1pt % Reduce space between items
\setcounter{enumi}{12}
\item  Yi Yang, {\bf Wenjie Hu} and Guohong Cao, ``Energy-Aware CPU Frequency Scaling for Mobile Video Streaming'', {\em IEEE Transactions on Mobile Computing (TMC)}, 2018.
\item Shan Yang, Jay Pillai, {\bf Wenjie Hu} and Bo Moon, ``Estimating Lift In Total Revenue From Microsoft Rewards'', {\em MSJAR}, 2018.
\item {\bf Wenjie Hu} and Guohong Cao, ``Quality-Aware Traffic Offloading in Wireless Networks'', {\em IEEE Transactions on Mobile Computing (TMC)}, 2017.
\item Yibo Wu, Yi Wang, {\bf Wenjie Hu} and Guohong Cao, ``SmartPhoto: A Resource-Aware Crowdsourcing Approach for Image Sensing with Smartphones'', {\em IEEE Transactions on Mobile Computing (TMC)}, 2016. %Citation: 1.
\item Bo Zhao, {\bf Wenjie Hu}, Qiang Zheng, and Guohong Cao, ``Energy-Aware Web Browsing on Smartphones'', {\em IEEE Transactions on Parallel and Distributed Systems (TPDS)}, 2015. %Citation: 29, Independent Citation: 24.
\item Yong Cui, {\bf Wenjie Hu}, Hongyi Wang, ``Probabilistic Multi-path Routing for Multimedia over Wireless Mesh Networks'', {\em Ad Hoc \& Sensor Wireless Networks(AHWSN)}, 2010.
\item Wei Gao, {\bf Wenjie Hu}, and Guohong Cao, ``Interest-Based Data Dissemination in Opportunistic Mobile Networks: Design, Implementation and Evaluation'', {\em Opportunistic Mobile Social Networks}, 2014. %Citation: 2, Independent Citation: 2.
\end{enumerate}

{\bf Patent}
\vspace*{.05in}
\begin{enumerate} [leftmargin=7mm]%\itemsep -1pt % Reduce space between items
\setcounter{enumi}{19}
\item Shan Yang, Jay Pillai, {\bf Wenjie Hu} and Bo Moon, ``Computing Resource-efficient, machine learning-based techniques for measuring an effect of participation in an activity'', submitted.
\end{enumerate}

%{\bf Book Chapter}
%\vspace*{.05in}
%\begin{enumerate} [leftmargin=5mm]%\itemsep -1pt % Reduce space between items
%\setcounter{enumi}{16}
%
%\end{enumerate}



%
%\section{\sc Project Experience}
%{\bf Pervasive Data Access System on Mobile Devices} \small \texttt{(System development, Data mining, Java)}
%
%\vspace*{-.15in}
%Built a server to crawl news from CNN, and a mobile app for users to obtain news either from the server or from neighbors according to their interests. Predicted users' interests using the Latent Dirichlet Allocation (LDA) model based on their reading history. Developed theoretical framework to analyze the energy-delay tradeoffs.
%
%{\bf Popular Topic Extractor} \small \texttt{(Distributed system, Text mining, C/C++)}
%
%\vspace*{-.15in}
%Developed a MapReduce framework to extract the most popular topics on a group of Facebook webpages (10GB+) using a group of computers. Deployed two-phase commit process to avoid system fault.
%
%{\bf U.S. Census Return Rate Challenge} \small \texttt{(Data mining, R)}
%
%\vspace*{-.15in}
%Developed statistical models to predict census mail return rate using RadomForest based on the demographic characteristics.
%
%{\bf User Retention Analysis on MOOCs} \small \texttt{(Data mining, Python)}
%
%\vspace*{-.15in}
%Analyzed a large scale of Coursera trace (70GB+) to study the user retention behavior. Built models to predict users who may finish all the courses.
%
%{\bf Traffic Aggregation Framework} \small \texttt{(Distributed system, Java)}
%
%\vspace*{-.15in}
%Deployed a collaboration service among a group of mobile devices to elect one with the best cellular signal to forward all the network traffic to save energy and reduce delay. Analyzed the cost-benefit tradeoffs. Wrapped the whole service together and provided simple APIs for other developers.

%\section{\sc Industrial Experience}
%{\bf Research Intern}\\
%Samsung Research America, San Jose, CA \hfill Summer 2014
%
%\vspace*{-.15in}
%Developed a system to improve the driving safety by leveraging the vehicle-to-vehicle communication. Designed a new MAC protocol to reduce the message delivery latency.


%\section{\sc Public Speaking}
%{\bf Invited Talks}
%
%%\vspace*{-.15in}
%\begin{enumerate} [leftmargin=5mm]% \itemsep -1pt % Reduce space between items
%\item Energy Optimization for Smartphones in Wireless Networks, {\em Peking University}, Beijing, China, May 2015.
%\item Energy Efficient Data Access on Mobile Devices, {\em Tsinghua University}, Beijing, China, May 2015.
%\item Cross-layer Design in Wireless Mesh Networks, {\em the First Asia-FI (Future Internet)}, Seoul, South Korea, 2008.
%\end{enumerate}
%\vspace*{-0.1in}
%{\bf Conference Presentations}
%%\vspace*{-.15in}
%\begin{enumerate}[leftmargin=5mm]
%\setcounter{enumi}{3}
%\item Energy-Aware Video Streaming on Smartphones, {\em the 34th IEEE International Conference on Computer Communications (INFOCOM)}, Hong Kong, China, April 2015.
%\item Quality-Aware Traffic Offloading in Wireless Networks, {\em the 15th ACM International Symposium on Mobile Ad Hoc Networking and Computing (MobiHoc)}, Philadelphia, PA, August 2014.
%\item SmartPhoto: A Resource-Aware Crowdsourcing Approach for Image Sensing with Smartphones, {\em the 15th ACM International Symposium on Mobile Ad Hoc Networking and Computing (MobiHoc)}, Philadelphia, PA, August 2014.
%\item Energy Optimization Through Traffic Aggregation
% in Wireless Networks, {\em the 33rd IEEE International Conference on Computer Communications (INFOCOM)}, Toronto, Canada, April 2014.
%\item Forwarding Redundancy in Opportunistic Mobile Networks: Investigation and Elimination, {\em the 33rd IEEE International Conference on Computer Communications (INFOCOM)}, Toronto, Canada, April 2014.
%\item Mobility-Assisted Energy-Aware User Contact Detection in Mobile Social Networks, {\em the 33rd IEEE International Conference on Distributed Computing Systems (ICDCS)}, Philadelphia, PA, July 2013.
%
%\end{enumerate}


%\section{\sc Teaching and Mentoring Experience}
%{\bf Teaching:}\\
%\vspace{-3mm}
%\begin{list2}
% \item {\bf Teaching Assistant}, CMPSC 311: Introduction to Systems Programming \hfill Fall 2010\\
%     Dept. of Computer Science and Engineering, Pennsylvania State University
% \item {\bf Teaching Assistant}, Wireless Communication and Mobile Computing \hfill Fall 2008\\
%     Dept. of Computer Science and Technology, Tsinghua University
%\end{list2}
%
%
%{\bf Mentoring:}
%\begin{list2}
% \item {\bf Undergraduates:}\\
%    - Max Freilich, project on developing pervasive data access systems\\
%    - Bryan Dickens, project on developing sound sensing systems\\
%    - Kevin-Michael Sheeran, project on extending video streaming modules on VLC
% \item {\bf Master students:}\\
%     - Ajay Harshavardhan, research on energy optimization in WiFi networks
% \item {\bf Ph.D students:}\\
%    - Yeli Geng, research on mobile cloud computing\\
%    - Yibo Wu, research on crowdsourcing based image sensing\\
%    - Yi Yang, project on measuring energy of mobile Ads\\
%    - Junpeng Qiu, research on wearable devices\\
%\end{list2}
%\section{\sc Travel History}
%South Korea \hfill 2/17/2008-2/24/2008\\
%Canada \hfill 4/27/2014-5/2/2014

%\section{\sc Professional Experience}
%%{\bf \em Summary: 49 reviews (30 journal reviews, 19 conference reviews)}\\
%
%\vspace{-0.15in}
%{\bf Professional Service}\\
%{\em Program Committee}, International conference on Computers, Data Management and Technology Applications, 2017\\
%{\em Program Committee}, International Conference on Information \& Communication Technology and Systems, 2016\\
%{\em Session Chair}, IEEE International Conference on Computer Communications (INFOCOM), 2015\\
%{\em Program Committee}, International Conference on Computer Applications \& Technology, 2015, 2017\\
%{\em Program Committee},  International Conference on Mechanics \& Applied Physics, 2015\\
%
%\vspace{-0.15in}
%{\bf Journal Reviewer}\\
%IEEE/ACM Transactions on Networking 2017, 2016, 2015 \\
%IEEE Transactions on Vehicular Technology 2016, 2015\\
%IEEE Transactions on Mobile Computing 2017, 2016, 2015 \\
%Mobile Networks and Applications 2016 \\
%Elsevier Computer Communications 2016, 2015\\
%Future Generation Computer Systems 2016 \\
%Wireless Networks 2017, 2016\\
%ACM Transactions on Sensor Networks 2016 \\
%IEEE Journal on Selected Areas in Communications (JSAC) 2016, 2015, 2014\\
%Mobile Information Systems 2015 \\
%Journal of Parallel and Distributed Computing 2015 \\
%IEEE ACCESS 2015
%
%{\bf Conference Reviewer}\\
%International Conference on Information \& Communication Technology and Systems 2016 \\
%International Conference on Computer Communication and Networks 2016 \\
%International Conference on Computing, Networking and Communications (ICNC) 2016\\
%IEEE International Symposium on Multimedia (ISM) 2015\\
%IEEE/IFIP International Conference on Dependable Systems and Networks 2015 \\
%International Conference on Information Technology (ICIT) 2015\\
%IEEE International Conference on Smart Grid Communications 2015 \\
%IEEE Global Communications Conference 2015\\
%IEEE International Conference on Computer Communications (INFOCOM) 2015\\
%IEEE International Conference on Communications (ICC) 2015\\
%International Conference on Computer Communications and Networks (ICCCN) 2015\\
%IEEE International Conference on Network Protocols (ICNP) 2014\\
%IEEE Military Communications Conference (MILCOM) 2014\\
%IEEE International Workshop on Quality of Service 2010\\
%IP Mobility Management and Architecture (IPMMA) 2008

%\section{\sc References}
%{\bf Guohong Cao}\\
%Professor, IEEE Fellow, Dept. of Computer Science and Engineering\\
%The Pennsylvania State University\\
%{\color{magenta} gcao@cse.psu.edu}\\
%1-814-863-1241\\
%
%\vspace{1mm}
%{\bf Sencun Zhu}\\
%Associate Professor, Dept. of Computer Science and Engineering\\
%The Pennsylvania State University\\
%{\color{magenta} szhu@cse.psu.edu}\\
%1-814-865-0995\\
%
%\vspace{1mm}
%{\bf Chiu Ngo}\\
%Doctor, Senior Director, Advanced Technology Lab\\
%Samsung Research America\\
%{\color{magenta} chiu.ngo@samsung.com} \\



\end{resume}
\end{document}




